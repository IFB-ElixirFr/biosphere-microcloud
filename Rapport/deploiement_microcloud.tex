\chapter{Déployer une instance MicroCloud et l'utiliser}

\section{Déployer l'appliance MicroCloud}

L'appliance MicroCloud est disponible dans le catalogue RAINBio \url{https://biosphere.france-bioinformatique.fr/catalogue/appliance/150/}. Pour déployer l'appliance, cliquer sur \textbf{Lancer} ou \textbf{Déploiement avancé}.

Les déploiements se font sur le cloud prabi-girofle sur lequel est installé la \hyperref[VM permanente]{VM permanente}.
Un déploiement prend environ 40 minutes.
Il vaut mieux en lancer 2 en même temps afin d'être sûr d'en avoir au moins un de fonctionnel.
Les échecs dûs au paquet \textbf{procps} sont assez fréquents.

\section{Insérer un nouvel organisme (partie Syntactic)}
Il faut avoir créé au préalable un utilisateur admin pour valider la prestation. En cela le script create\_user.sh peut servir.
Il est dans le répertoire \path{/var/tmp/slipstream/biosphere-microcloud/master}
\bigskip

Nous avons également besoin des données de taxonomie pour ce nouvel organisme. Pour cela, il faut mettre à jour la base TAXONOMYDB de la VM permanente :
\begin{itemize}
	\item en copiant les données de prod de TAXONOMYDB et pkgdb.O\_Taxonomy pour cet organisme
	\item ou en déployant les WF TAXONOMYDB et TAXONOMY
\end{itemize}
\bigskip

Pour insérer un nouvel organisme, il est possible d’utiliser le module micPrestation.
La procédure à suivre est la suivante :
\begin{itemize}
	\item Télécharger un fasta et décompresser l'archive.
	\item Remplir le fichier Template\_Prestation.csv qui trouve dans le répertoire \path{/env/cns/proj/agc/module/products/micsyntactic/unix-noarch/lib}
	\item Lancer la création de la prestation
\end{itemize}

\begin{lstlisting}[style=bash]
prestaBatch.py --csv Template_Prestation.csv -sd ${working_dir} -A ${Annotator_name} -T genome
\end{lstlisting}

Il y a un exemple de fichier Template\_Prestation.csv dans le dossier prestation : \path{/env/cns/wwwext/html/agc/ftp/MicroCloud/prestation}.
\newline
\begin{mycolorbox}
	Il faut un utilisateur admin pour accepter la prestation depuis l'interface web (voir le fichier : \url{https://github.com/IFB-ElixirFr/biosphere-microcloud/blob/master/master/create_user.sh}).
\end{mycolorbox}
\bigskip

Pour valider la prestation sans passer par le web faire :
\begin{lstlisting}[style=SQL]
USE PRESTATIONDB;
SELECT * FROM Prestation;
UPDATE Prestation SET PS_status='accepted' WHERE PS_status='inprogress';
\end{lstlisting}
\bigskip
Puis, pour insérer la nouvelle séquence en base, utiliser le module micGenome :
\begin{lstlisting}[style=bash]
loadGenomeNS.sh -p ${PS_id} -o ${O_id}
\end{lstlisting}

\begin{mycolorbox}
	Une fois la prestation validée, reste le problème que les données des banques nécessaires pour le script \textbf{loadGenomeNS.sh} ne sont pas stockées dans la VM permanente actuellement (données des banques gérées par CABRI).
	Cette étape ne fonctionne donc pas.
\end{mycolorbox}

Le module micPrestation n'est plus à jour. Le schéma de la base TAXONOMYDB ayant été mis à jour depuis la MEP 3.14.0, il faudra modifier les scripts du module et supprimer les variables faisant référence aux colonnes O\_Gram, O\_genetic\_code et level.
\newline

\section{Déployer un WF}
Les sources sont dans \path{/env/cns/proj/agc/tools/COMMON/JBPMmicroscope/jbpmmicroscope}
\newline

Répertoire de travail Tomcat : \path{/env/cns/proj/agc/tools/COMMON/JBPMmicroscope/tomcat}
\newline

Emplacement des modules : \path{/env/cns/proj/agc/module/products}
\newline

\begin{mycolorbox}
	Si les données ont été copiées depuis la prod et non pas insérées avec micPrestation, il faut, dans la table pkgdb.Sequence, vérifier que le status des S\_id existants soit 'inFunctional' et non 'inProduction' pour que les WF prennent bien en compte les nouveaux S\_id.
\end{mycolorbox}

\begin{lstlisting}[style=SQL]
USE pkgdb;
UPDATE Sequence SET S_status='inFunctional' WHERE S_status='inProduction';
\end{lstlisting}

\subsection{Déployer le WF DIRECTON}
Modules requis : \textbf{bagsub-2.4.3}, \textbf{AGCScriptToolMic-2.0}, \textbf{micGenome-7.0.0}, \textbf{micJBPMwrapper-3.10.8} et \textbf{micPrestation-2.0} et \textbf{micDirecton-1.0}.

\begin{lstlisting}[style=bash]
#### Deploy DIRECTON WF ####
cd /env/cns/proj/agc/tools/COMMON/JBPMmicroscope/tomcat/

# Deploy WF
JBPMmicroscope deployProcess -dirXMLSrc ../jbpmmicroscope/src/main/process-definitions/jpdl/BagSub/ -defNames DIRECTON

# Deploy and start CRON
JBPMmicroscope deployProcess -dirXMLSrc ../jbpmmicroscope/src/main/process-definitions/jpdl -defNames CRON_DIRECTON
JBPMmicroscope startCron -names CRON_DIRECTON

# Check CRON and WF status
JBPMmicroscope showProcessDefinitions

# Force run
JBPMmicroscope signalCron -names CRON_DIRECTON -signal forceRun; JBPMmicroscope signalProcess -pid 1

JBPMmicroscope showProcessInstances

# Take a look at running processes
tomcat_logs

# relaunch WF if status is sendmail
JBPMmicroscope relaunchAnalysis -pidCron 1 -pidWF 2 -signal forceRun

# Check tables pkgdb.Workflow_Jobs, pkgdb.Directon and pkgdb.GO_Directon_CPD
mysqlagcdb
\end{lstlisting}
