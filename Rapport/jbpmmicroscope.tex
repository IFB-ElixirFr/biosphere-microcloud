\section{Installation de jbpmmicroscope}

Dans cette section JBPMmicroscope peut faire référence à la base de données JBPMmicroscope ou bien à la commande JBPMmicroscope qui permet le déploiement des WF.
\newline

Il faut générer les fichiers suivants :
\begin{itemize}
    \item \textbf{jbpmmicroscope.jar}
    \item \textbf{SystemActorsLauncher.jar}
    \item \textbf{jbpmmicroscope-server.war}
\end{itemize}
\bigskip

Puis mettre à jour le fichier \textbf{~/.m2/genosphere-settings.xml} :
\begin{itemize}
    \item  Mettre à jour le local path
    \item Supprimer/modifier le login et le mot de passe
\end{itemize}

Pour créer les .jar dans le répertoire \textbf{jbpmmicroscope-client/target/} faire :

\begin{lstlisting}[style=bash]
mvn -Pconfig-microcloud -s ~/.m2/genosphere-settings.xml clean install
\end{lstlisting}
\bigskip

Il faut ajouter les sources (pour le déploiement des WF) et le script JBPMmicroscope
\begin{lstlisting}[style=bash]
jar uf jbpmmicroscope-client/target/jbpmmicroscope.jar -C ./ src/
jar uf jbpmmicroscope-client/target/jbpmmicroscope.jar -C ./ JBPMmicroscope
\end{lstlisting}
\bigskip

Pour créer l'archive \textbf{jbpmmicroscope-server.war} dans le répertoire \textbf{jbpmmicroscope-server/target/} :
\begin{lstlisting}[style=bash]
mvn -Pconfig-microcloud -s ~/.m2/genosphere-settings.xml package
\end{lstlisting}
\bigskip

Copier le .war et les .jar sur etna0 dans le répertoire \textbf{/env/cns/wwwext/html/agc/ftp/MicroCloud/}
La VM master importe ces fichiers à l’aide des liens symboliques.
\newline

Le schéma de la base JBPMmicroscope est créé automatiquement par hibernate (\textbf{hibernate.cfg.xml} voir le paramètre hbm2ddl.auto) lors de la première exécution d'une commande JBPMmicroscope.
\newline

Les identifiants de connexion à la base JBPMmicroscope (hibernate) sont les suivants :
login: jbpm et password: jbpm. Ces identifiants peuvent être modifiés dans le composant master : paramètres \textbf{jbpm\_user} et \textbf{jbpm\_password} (penser à mettre à jour le fichier \textbf{hibernate.cfg.xml}, régénérer les .jar et le .war et les copier dans \textbf{/env/cns/wwwext/html/agc/ftp/MicroCloud}).
\newline

jbpmmicroscope nécessite un serveur Tomcat pour fonctionner. Tomcat 9 est installé dans le composant master : \textbf{/env/cns/proj/agc/tools/COMMON/JBPMmicroscope/tomcat}. Les identifiants tomcat sont les suivants :
login: \textbf{tomcat} et password: \textbf{tomcat}. Ces identifiants peuvent être modifiés dans le composant master : paramètres \textbf{tomcat\_user} et \textbf{tomcat\_password}.