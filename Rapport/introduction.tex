\section{Introduction}

Ce document présente le projet MicroCloud.
Il vise en particulier à donner une vue d'ensemble des choix effectués
et de leur implémentation ainsi qu'à fournir un guide utilisateur.

\begin{mycolorbox}
	La technologie SlipStream n'est plus maintenue et est obsolète.
	Le service Nuvla a pris fin le 15 mai 2020 peu de temps après la fin du projet.
	Le rapport présente ce qui fonctionnait à la fin du mois d'avril 2020.
	Ainsi, une partie des informations de ce rapport (liens vers les composants Nuvla) n'est plus pertinente.\\
	L'IFB a déployé sa propre instance de SlipStream mais nous n'avons pas accès à un équivalent de Nuvla pour modifier les composants.
	De plus, à l'heure actuelle, le déploiement de l'application MicroCloud depuis le catalogie RAINBio ne fonctionne pas.\\
	L'IFB est en train d'étudier de nouvelles solutions comme la solution Terraform Cloud en remplacement (pas de délais annoncés pour le moment, dépendra des ressources humaines disponibles). Quelle que soit la solution retenue, le mode de fonctionnement basé sur des scripts dans un dépôt git restera le même.
\end{mycolorbox}

\subsection{Objectifs du projet MicroCloud}

L'objectif du projet est de pouvoir déployer une instance de MicroScope dans le cloud.

\subsection{Documentation et code du projet}

\begin{itemize}
	\item Projet GitHub biosphere-microcloud : \url{https://github.com/IFB-ElixirFr/biosphere-microcloud}
	\item Projet GitHub biosphere-commons : \url{https://github.com/IFB-ElixirFr/biosphere-commons}
	\item Application MicroCloud : \url{https://nuv.la/module/ifb/devzone/MicroCloud/}
	\item Catalogue RAINBio : \url{https://biosphere.france-bioinformatique.fr/catalogue/}
	\item Appliance MicroCloud : \url{https://biosphere.france-bioinformatique.fr/catalogue/appliance/150/}
	\item Redmine du projet : \url{https://intranet.genoscope.cns.fr/agc/redmine/projects/microcloud} 
	\item Documents sur le cloud IFB: voir le wiki du projet Redmine et dans \path{/env/cns/proj/bureautique/agc/Formations_Suivies/20180619-Formation-Cloud-IFB}
\end{itemize}

\subsection{Contacts importants}

\begin{itemize}
	\item Contact général: Mathieu Dubois (\href{mailto:mdubois@genoscope.cns.fr}{mdubois@genoscope.cns.fr}), Genoscope
	\item Dépôt \href{https://github.com/IFB-ElixirFr/biosphere-commons}{biosphere-commons}: Christophe Blanchet (\href{mailto:Christophe.BLANCHET@france-bioinformatique.fr}{Christophe.BLANCHET@france-bioinformatique.fr}), PRABI UCBL1 Lyon
	\item Cloud prabi-girofle et problèmes d'accès à la VM permanente: Stéphane Delmotte (\href{mailto:Stéphane.Delmotte@univ-lyon1.fr}{Stéphane.Delmotte@univ-lyon1.fr}), PRABI UCBL1 Lyon
	\item Installation du serveur mysql dans la VM permanente: Sylvain Bonneval (\href{mailto:bonneval@genoscope.cns.fr}{bonneval@genoscope.cns.fr}), Genoscope
	\item Ouverture des ports pour accéder à la VM permanente: Guy Perrière (\href{mailto:guy.perriere@univ-lyon1.fr}{guy.perriere@univ-lyon1.fr}), Directeur du PRABI UCBL1 Lyon
\end{itemize}
