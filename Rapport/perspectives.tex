\chapter{Bilan et perspectives}

\section{Bilan}

\subsection{Un prototype fonctionnel}

Le projet a permis de déployer une instance de MicroScope sur le cloud IFB
comprenant la partie web, les BD et les WF de calcul.
Le prototype permet faire tourner un WF simple (DIRECTON).

Pour cela, nous avons réalisé un script d'installation de MicroScope.
Ainsi, nous avons pu mieux comprendre les interactions entre les composants.
Ceci a été en particulier utile pour le composant \component{jbpmmicroscope} dont l'installation et la configuration
se sont avérées très délicates.

\todo[inline]{Il faut sans doute rajouter des choses dans cette partie.}

\subsection{Prise en main de BioMAJ et améliorations}

Un des objectifs du projet était d'utiliser \href{https://biomaj.genouest.org/}{BioMAJ}
pour la mise à jour des banques dans le cloud.

Le projet nous a permis de prendre en main le logiciel ce qui est intéressant car
il est utilisé dans le projet PanGBank.
De plus, il est envisagé pour la copie des banques au Genoscope en remplaçant de Cabri.

Au cours du projet, nous avons ajouté plusieurs fonctionnalités à BioMAJ:
\begin{itemize}
	\item Utilisation des hardlinks lors de la mise à jour d'une banque (ce qui est plus rapide et consomme moins de place).
	\item Ajout d'options SSL (pour configurer l'accès aux ressources FTPS/HTTPS).
	\item Refactoring des downloaders (nettoyage du code).
	\item Possibilité de configurer le comportement en cas d'échec du téléchargement (combien de temps attendre et combien de fois ré-essayer).
\end{itemize}

Ce travail bénéficie à toute la communauté.

\subsection{Identification de limites dans l'architecture de MicroScope}

\todo[inline]{Fonctionnement sans données, frontières entre composants peu claires, dépendances.}

\subsection{Identification de limites de l'architecture cloud}

Le projet a aussi permis d'identifier un certains nombre de limites des architectures cloud.
Ces limites concernent principalement le stockage.

En effet, on ne veut pas reconstituer les banques pour chaque instance de MicroCloud.
Pour cela, nous  avons déployé une VM permanente (voir section \ref{VMpermanente}) dans le cloud \cloudInstance{ifb-prabi-girofle}
qui héberge les BD MySQL représentant les banques.
Cependant, nous n'avons pas suffisamment de place pour télécharger toutes les banques de MicroScope sur la VM permanente.
À l'heure actuelle, seule la banque UNIPROTKBDB est disponible.

Une solution partielle est le stockage permanent et partagé mis en place par l'IFB
qui est utilisé pour mettre à disposition les banques sur toutes les VM (ces banques sont mises à jour avec BioMAJ).
On pourrait alors utiliser les WF de mise à jour des banques de MicroScope.
Cependant ceci n'est pas encore disponible sur \cloudInstance{ifb-prabi-girofle} (sur lequel tourne la VM permanente).

\section{Limites}

Outre les limites évoquées plus haut, le prototype a plusieurs limites:
\begin{itemize}
	\item La version de MicroScope n'est pas fixée: les scripts d'installation incluent la dernière version disponible sur les serveurs du Genoscope.
	\item À cause de ces limites, MicroCloud ne tourne que sur le cloud \cloudInstance{ifb-prabi-girofle}.
	\item Du fait que les banques sont mises-à-jour manuellement sur la VM permanente, on peut avoir un décalage entre la version de MicroScope déployée dans le cloud
	et la version des banques sur la VM.
\end{itemize}

\section{Améliorations}

Les principales améliorations au prototype sont:
\begin{itemize}
	\item Résoudre le problèmes des banques. Ceci passe sans doute par l'améliorations de l'infrastructure de Biosphere.
	\item Améliorer l'installation des dépendances pour les WF.
\end{itemize}
\bigskip

D'un point de vue technique, les améliorations sont:
\begin{itemize}
    \item Améliorations du côté de biosphère-commons (option -e).
    \item Certains scripts sont sourcés avec une option (ce qui est valide en bash mais n'est pas portable).
    \item Paralléliser les scripts pour diminuer le temps de déploiement (en particulier pouvoir faire le maximum de choses avant d'attendre les signaux des autres composants).
\end{itemize}
