\chapter{Bilan et perspectives}

\section{Bilan}

Le projet a globalement atteint ses objectifs car il permet de déployer une instance de MicroScope dans le cloud IFB (cf.~\autoref{subsec:prototype}).

De plus, le travail sur ce projet a permis d'identifier des limites (au niveau de l'architecture ou du code) dans MicroScope (cf.~\autoref{subsec:limites_microscope})
et dans les architectures cloud (cf.~\autoref{subsec:limites_coud})
qui complexifient l'installation de MicroScope.

Un des effets de bord du projet est que nous avons travaillé avec et sur \project{BioMAJ} (cf.~\autoref{subsec:biomaj}).

\subsection{Le prototype fonctionnel} \label{subsec:prototype}

Le projet permet de déployer une instance de MicroScope
sur le cloud IFB.
Le prototype permet faire tourner un WF simple (DIRECTON).
Cependant, l'insertion des séquences est problématique (cf~\autoref{sec:syntactic}).
Les instructions pour le déployer sont dans le~\autoref{chap:deploiement}.
Pour des raisons techniques expliquées plus bas,
MicroCloud ne peut être déployé que sur le cloud \cloudInstance{ifb-prabi-girofle}.

Pour cela, nous avons réalisé une procédure d'installation de MicroScope.
Ainsi, nous avons pu mieux comprendre les interactions entre les composants.
Ceci a été en particulier utile pour le composant \component{jbpmmicroscope} dont l'installation et la configuration
se sont avérées très délicates (voir~\autoref{sec:installation_jbpmmicroscope}).

\subsection{Identification de limites dans MicroScope} \label{subsec:limites_microscope}

Au cours du projet, nous avons rencontré des difficultés liées à l'architecture ou au code de MicroScope.
Les principales limites rencontrées sont:
\begin{enumerate}
    \item \emph{Fonctionnement sans données.} La première difficulté est qu'il est très difficile (voire impossible) de faire tourner MicroScope sans au moins une séquence.
          Ceci est lié au fait que beaucoup de pages font appel à des fonctions qui utilisent la séquence stockée dans la session.
    \item \emph{Frontières entre les composants peu claires.} Une difficulté pour l'installation est que MicroScope est organisé en couches (DB, web, worfklow) et non en modules.
          Ainsi, si on installe \component{micJBPMwrapper}, on a tous les WF mais il faut alors créer toutes les bases.
          On ne peut pas installer juste un WF.
    \item \emph{Dépendances.} MicroScope a de nombreuses dépendances mais elles ne sont pas listées.
          De plus, certains composants ne sont pas publiés (\module{micGenome}, \module{micPrestation})
          et/ou ne sont pas facilement installables (\component{jbpmmicroscope}).
\end{enumerate}

\subsection{Identification de limites de l'architecture cloud} \label{subsec:limites_coud}

Le projet a aussi permis d'identifier un certain nombre de limites des architectures cloud
qui complexifient l'installation de MicroScope.
Ces limites concernent principalement le stockage.
En effet, d'une part, le stockage disponible sur chaque VM est limité et
d'autre part n'est pas permanent.

Or MicroScope a besoin des données des banques (sous forme de fichiers et de bases SQL) pour fonctionner.
D'une part, ceci représente plusieurs dizaines de Go
et nécessite des traitements pouvant être longs.
Ainsi, il n'est pas possible de constituer ces banques pour chaque déploiement.

De plus, les données stockées dans une instance sont perdues lorsqu'on la supprime.

Une solution partielle au premier problème passe par l'utilisation d'une VM permanente
qui héberge les BD MySQL représentant les banques (voir section \ref{sec:acces_banques}).
Pour des raisons techniques, seules les VM déployées dans le cloud \cloudInstance{ifb-prabi-girofle}
peuvent accéder à cette VM.
Ainsi MicroCloud ne peut être déployé que sur le cloud \cloudInstance{ifb-prabi-girofle}.

Le deuxième problème n'a pas été abordé.

\begin{warningbox}
    L'ouverture de la VM est à la discrétion du directeur du PRABI (voir~\autoref{sec:contacts}).
    Elle est à renouveler tous les ans (le dernier renouvellement d'une durée d'un an date du 2020-01-27).
\end{warningbox}

\subsection{Prise en main de \project{BioMAJ} et améliorations}\label{subsec:biomaj}

Un des objectifs du projet était d'utiliser \project{BioMAJ} pour la mise à jour des banques dans le cloud.

Bien que cette partie n'ait pas été abordée, le projet nous a permis de prendre en main le logiciel
ce qui est intéressant car il est utilisé dans le projet PanGBank.
De plus, il est envisagé en remplaçant de CABRI pour la copie des banques au Genoscope.

Au cours du projet, nous avons ajouté plusieurs fonctionnalités à \project{BioMAJ}:
\begin{itemize}
    \item Utilisation des hardlinks lors de la mise à jour d'une banque (ce qui est plus rapide et consomme moins de place).
    \item Ajout d'options SSL (pour configurer l'accès aux ressources FTPS/HTTPS).
    \item Refactoring des downloaders (nettoyage du code).
    \item Possibilité de configurer le comportement en cas d'échec du téléchargement (temps d'attente entre les essais, nombre d'essais).
\end{itemize}

Ce travail bénéficie à toute la communauté.

\section{Limites}

Outre les limites évoquées plus haut, le prototype a plusieurs limites:
\begin{itemize}
    \item Les scripts de déploiement utilisent la dernière version des composants disponible sur les serveurs du Genoscope.
          Donc la version de MicroScope n'est pas fixée dans les composants cloud.
          De plus, si une nouvelle version de MicroScope change la procédure d'installation,
          il faut indépendamment mettre à jour le script d'installation dans \project{biosphere-microcloud}.
    \item Comme expliqué plus haut, MicroCloud ne tourne que le cloud \cloudInstance{ifb-prabi-girofle}.
    \item Du fait que les banques sont mises à jour manuellement sur la VM permanente, on peut avoir un décalage entre la version de MicroScope déployée dans le cloud
          et la version des banques sur la VM.
\end{itemize}

\section{Améliorations}

Les principales améliorations au prototype sont:
\begin{itemize}
    \item Résoudre le problème des banques.
          Ceci passe sans doute par l'amélioration de l'infrastructure de Biosphere.
\end{itemize}


D'un point de vue technique, les améliorations sont:
\begin{itemize}
    \item Certains déploiements échouent.
          Il semble que ces problèmes soient dus à l'installation du paquet \textbf{procps}.
          Il vaut mieux en lancer 2 en même temps afin d'être sûr d'en avoir au moins un de fonctionnel.
    \item Améliorer l'installation des logiciels métier pour les WF:
          la solution que nous avons adoptée (voir~\autoref{sec:installation_logiciels_metier}) est très \textit{ad-hoc}
          car la structure des dépendances fait qu'on doit installer tous les logiciels métier;
          ceci nécessite sans doute un travail dans \component{micJBPMwrapper} pour que les WF chargent leurs propres dépendances
          \footnote{Ceci est en cours (voir \href{https://intranet.genoscope.cns.fr/agc/redmine/issues/7830}{\#7830}).}.
    \item Améliorations du côté de \project{biosphère-commons} (option -e).
    \item Certains scripts sont sourcés avec une option (ce qui est valide en bash mais n'est pas portable).
    \item Paralléliser les scripts pour diminuer le temps de déploiement (en particulier pouvoir faire le maximum de choses avant d'attendre les signaux des autres composants).
    \item L'insertion de séquences ne fonctionne pas complètement (voir~\autoref{sec:syntactic}).
          Cependant, la copie d'un organisme reste possible (voir~\autoref{sec:nouvelle_donne_organisme}).
    \item Le module \module{micPrestation} (utilisé pour insérer des organismes sans faire de prestation)
          ne fonctionne pas suite aux modifications du schéma de TAXONOMYDB réalisée dans \href{https://intranet.genoscope.cns.fr/agc/redmine/versions/142}{MicroScope Web 3.14.0}.
          Il faudra modifier les scripts du module et supprimer les références aux colonnes \DB{O\_Gram}, \DB{O\_genetic\_code} et \DB{level}.
\end{itemize}

Le code de déploiement des VM contient une liste d'améliorations (voir le fichier \filename{README.md} dans le répertoire de chaque composant sur le dépôt \project{biosphere-microcloud}).

Enfin, nous avions prévu de faire de tests de performance (performance des montages NFS, des requêtes SQL, \textit{etc}.)
à la fin du projet
mais à cause des problèmes sur le cloud IFB mentionnés au début de ce rapport,
nous n'avons pas pu les faire.
