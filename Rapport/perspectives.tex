\chapter{Bilan et perspectives}

\section{Bilan}

Le projet a globalement atteint ses objectifs car il permet de déployer une instance de MicroScope dans le cloud IFB (cf.~\autoref{subsec:prototype}).

De plus, le travail sur ce projet a permis d'identifier des limites (au niveau de l'architecture ou du code) dans MicroScope (cf.~\autoref{subsec:limites_microscope})
et dans les architectures cloud (cf.~\autoref{subsec:limites_coud}).

Un des effets de bord du projet est que nous avons travaillé avec et sur \project{BioMAJ} (cf.~\autoref{subsec:biomaj}).

\subsection{Le prototype fonctionnel} \label{subsec:prototype}

Le projet permet de déployer une instance de MicroScope
comprenant la partie web, les BD et les WF de calcul
sur le cloud IFB.
Le prototype permet faire tourner un WF simple (DIRECTON).
Cependant, l'insertion des séquences est problématique.
Les instructions pour le déployer sont dans le~\autoref{chap:deploiement}.

Pour cela, nous avons réalisé une procédure d'installation de MicroScope.
Ainsi, nous avons pu mieux comprendre les interactions entre les composants.
Ceci a été en particulier utile pour le composant \component{jbpmmicroscope} dont l'installation et la configuration
se sont avérées très délicates\todo{faire une section sur ça.}.

\subsection{Identification de limites dans MicroScope} \label{subsec:limites_microscope}

Les principales limites dans l'architecture et la code de MicroScope sont:
\begin{enumerate}
    \item \emph{Fonctionnement sans données:} La première limite est
          qu'il est très difficile (voire impossible) de faire tourner MicroScope sans au moins une séquence.
          Ceci est lié au fait que beaucoup de pages font appel à des fonctions qui utilisent la séquence stockée dans la session.
    \item \emph{Frontières entre les composants peu claires:} Une difficulté pour l'installation est que MicroScope est organisé en couches (DB, web, worfklow) et non en modules.
          Ainsi, si on installe \component{micJBPMwrapper}, on a tous les WF mais il faut alors créer toutes les bases.
          On ne peut pas installer juste un WF.
    \item \emph{Dépendances:} MicroScope a de nombreuses dépendances mais elles ne sont pas listées.
          De plus, certains composants ne sont pas publiés (\module{micGenome}, \module{micPrestation})
          et ne sont pas facilement installables (\component{jbpmmicroscope}).
\end{enumerate}

\subsection{Identification de limites de l'architecture cloud} \label{subsec:limites_coud}

Le projet a aussi permis d'identifier un certains nombre de limites des architectures cloud.
Ces limites concernent principalement le stockage.

En effet, on ne veut pas reconstituer les banques pour chaque instance de MicroCloud.
Pour cela, nous  avons déployé une VM permanente (voir section \ref{VMpermanente}) dans le cloud \cloudInstance{ifb-prabi-girofle}
qui héberge les BD MySQL représentant les banques.
Pour des raisons techniques, seules les VM déployées dans le cloud \cloudInstance{ifb-prabi-girofle}
peuvent accéder à cette VM
Ainsi MicroCloud ne peut être déployé que sur le cloud \cloudInstance{ifb-prabi-girofle}.\todo{Lien vers une section qui donne les détails techniques.}
Cependant, nous n'avons pas suffisamment de place pour télécharger toutes les banques de MicroScope sur la VM permanente.
À l'heure actuelle, seule la banque UNIPROTKBDB est disponible.

Une solution partielle est le stockage permanent et partagé mis en place par l'IFB
qui est utilisé pour mettre à disposition les banques sur toutes les VM (ces banques sont mises à jour avec \project{BioMAJ}).
On pourrait alors utiliser les WF de mise à jour des banques de MicroScope.
Cependant ceci n'est pas encore disponible sur \cloudInstance{ifb-prabi-girofle} (sur lequel tourne la VM permanente).

\begin{mycolorbox}
    L'ouverture de la VM est à la discrétion du directeur du PRABI (voir~\autoref{sec:contacts}).
    Elle à renouveler régulièrement (le dernier renouvellement d'une durée d'un an date du 2020-01-27).
\end{mycolorbox}

\subsection{Prise en main de \project{BioMAJ} et améliorations}\label{subsec:biomaj}

Un des objectifs du projet était d'utiliser \project{BioMAJ} pour la mise à jour des banques dans le cloud.

Bien que cette partie n'ait pas été abordée, le projet nous a permis de prendre en main le logiciel
ce qui est intéressant car il est utilisé dans le projet PanGBank.
De plus, il envisagé en remplaçant de CABRI pour la copie des banques au Genoscope.

Au cours du projet, nous avons ajouté plusieurs fonctionnalités à \project{BioMAJ}:
\begin{itemize}
    \item Utilisation des hardlinks lors de la mise à jour d'une banque (ce qui est plus rapide et consomme moins de place).
    \item Ajout d'options SSL (pour configurer l'accès aux ressources FTPS/HTTPS).
    \item Refactoring des downloaders (nettoyage du code).
    \item Possibilité de configurer le comportement en cas d'échec du téléchargement (temps d'attente entre les essais, nombre d'essais).
\end{itemize}

Ce travail bénéficie à toute la communauté.

\section{Limites}

Outre les limites évoquées plus haut, le prototype a plusieurs limites:
\begin{itemize}
    \item La version de MicroScope n'est pas fixée: les scripts d'installation incluent la dernière version disponible sur les serveurs du Genoscope.
    \item Comme expliqué plus haut, MicroCloud ne tourne que le cloud \cloudInstance{ifb-prabi-girofle}.
    \item Du fait que les banques sont mises-à-jour manuellement sur la VM permanente, on peut avoir un décalage entre la version de MicroScope déployée dans le cloud
          et la version des banques sur la VM.
\end{itemize}

\section{Améliorations}

Les principales améliorations au prototype sont:
\begin{itemize}
    \item Résoudre le problèmes des banques.
          Ceci passe sans doute par l'améliorations de l'infrastructure de Biosphere.
    \item Améliorer l'installation des dépendances pour les WF.
\end{itemize}


D'un point de vue technique, les améliorations sont:
\begin{itemize}
    \item Certains déploiements échouent.
          Il semble que ces problèmes soient dus à l'installation du paquet \textbf{procps}.
          Il vaut mieux en lancer 2 en même temps afin d'être sûr d'en avoir au moins un de fonctionnel.
    \item Améliorations du côté de \project{biosphère-commons} (option -e).
    \item Certains scripts sont sourcés avec une option (ce qui est valide en bash mais n'est pas portable).
    \item Paralléliser les scripts pour diminuer le temps de déploiement (en particulier pouvoir faire le maximum de choses avant d'attendre les signaux des autres composants).
    \item L'insertion de séquences ne fonctionne pas.
          \todo[inline]{Clarifier ça. Voir~\autoref{chap:deploiement}}
    \item Le module \module{micPrestation} (utilisé pour insérer des organismes sans faire de prestation) ne fonctionne pas suite aux modifications du schéma de TAXONOMYDB réalisée dans \href{https://intranet.genoscope.cns.fr/agc/redmine/versions/142}{MicroScope Web 3.14.0}.
          Il faudra modifier les scripts du module et supprimer les variables faisant référence aux colonnes O\_Gram, O\_genetic\_code et level.
\end{itemize}

Le code de déploiement des VM contient une liste d'améliorations (voir le fichier \filename{README.md} dans le répertoire de chaque composant sur le dépôt \project{biosphere-microcloud}).
