\chapter{Mettre à jour des composants pour MicroCloud} \label{chap:creer_nouvelle_version}

Ce chapitre présente les commandes utilisées pour créer une nouvelle version
des fichiers contenant le code et les données qui seront utilisés
par les scripts d'installation des VM pour déployer MicroScope.
Ces opérations doivent être effectuées au Genoscope.
Les détails techniques sont dans le~\autoref{chap:micwebdeploy}.
Le répertoire d'export des archives et autres fichiers sur etna0 est \path{/env/cns/wwwext/html/agc/ftp/MicroCloud/}.

Les composants nécessaires sont:
\begin{itemize}
    \item Le code web et les bases de données (schéma + données de base) de l'instance (voir~\autoref{sec:nouvelle_version_web})
          qui sont installés par la VM frontend.
    \item Les données d'un organisme (voir~\autoref{sec:nouvelle_donne_organisme})
          qui sont aussi utilisées par la VM frontend.
    \item Les modules à installer (voir~\autoref{sec:nouvelle_liste_modules}).
    \item \component{jbpmmicroscope} (voir~\autoref{sec:nouvelle_version_jbpmmicroscope}).
\end{itemize}

La plupart des ces opérations utilisent le module \micWEBdeployVer.
Pour le charger:
\begin{lstlisting}[style=bash]
$ module load micWEBdeploy/0-MicroCloud-1.0
\end{lstlisting}

\section{Créer une nouvelle release du code web} \label{sec:nouvelle_version_web}

Pour créer une nouvelle version du code du web et des bases de données, choisir la version de MicroScope à déployer puis lancer le script \script{microscopeRelease.py} :
\begin{lstlisting}[style=bash]
$ MICVERSION=3.14.0
$ microscopeRelease.py --version ${MICVERSION} --output /env/cns/wwwext/html/agc/ftp/MicroCloud/microcloud-${MICVERSION}.tar.gz
\end{lstlisting}

Ensuite, mettre à jour le lien symbolique pour qu'il pointe vers la nouvelle archive :
\begin{lstlisting}[style=bash]
$ cd /env/cns/wwwext/html/agc/ftp/MicroCloud
$ ln -s microcloud-${MICVERSION}.tar.gz microcloud-latest.tar.gz
\end{lstlisting}

\section{Copier les données d'un organisme} \label{sec:nouvelle_donne_organisme}

Comme MicroScope ne peut pas fonctionner sans données, nous avons crée un système
pour importer les données syntaxique d'un organisme.

Pour créer une archive des données d'un organisme, utiliser le script \script{microscopeCopyOid.py} de la manière suivante :
\begin{lstlisting}[style=bash]
$ Oid=(*\theOid{}*)
$ microscopeCopyOid.py --oid ${Oid} --output /env/cns/wwwext/html/agc/ftp/MicroCloud/microscope_${Oid}.tar.gz
\end{lstlisting}

\begin{mycolorbox}
    Pour le moment seules les données concernant l'Oid \theOid{} peuvent être récupérées avec le script microscopeCopyOid.py car certains organismes ont besoin de tables spécifiques de la base GO\_SPE (typiquement la table acineto\_KO pour \theOrg{}).
    C'est cet organisme qui est installé dans MicroCloud.
\end{mycolorbox}

Ensuite, mettre à jour le lien symbolique pour qu'il pointe vers la nouvelle archive :
\begin{lstlisting}[style=bash]
$ Oid=(*\theOid{}*)
$ cd /env/cns/wwwext/html/agc/ftp/MicroCloud
$ ln -s microscope_${Oid}.tar.gz microscope_${Oid}-latest.tar.gz
\end{lstlisting}

\section{Mettre à jour la liste des modules} \label{sec:nouvelle_liste_modules}

Lors du déploiement de l'instance, un certain nombre de modules sont installés.
Pour mettre à jour cette liste, il faut utiliser le script \script{createModulesTarball.py}.

En premier lieu, faire la liste de tous les modules nécessaires à la création de la nouvelle instance MicroCloud et utiliser
le script \script{createModulesTarball.py} pour générer les archives de ces modules:
\begin{lstlisting}[style=bash]
$ createModulesTarball.py --modules_list bagsub-2.4.3 AGCScriptToolMic-2.0 micGenome-7.0.0 micJBPMwrapper-3.10.8 micPrestation-2.0 micDirecton-1.0
\end{lstlisting}

Actuellement l'environnement de MicroCloud ne permet d'utiliser que les modules \module[2.4.3]{bagsub}, \module[2.0]{AGCScriptToolMic}, \module[7.0.0]{micGenome}, \module[3.10.8]{micJBPMwrapper}, \module[2.0]{micPrestation} et \module[1.0]{micDirecton}.

\begin{mycolorbox}
    Lors de tout ajout de nouveaux modules, il faut modifier le script de l'étape Depolyment de la VM master
    pour faut ajouter les variables d'environnement nécessaires aux les nouveaux modules
    dans le fichier \path{microcloud.profile}.
\end{mycolorbox}

\section{Mettre à jour \component{jbpmmicroscope}} \label{sec:nouvelle_version_jbpmmicroscope}

MicroCloud utilise une version de \component{jbpmmicroscope} compilée avec des options spéciales.

La configuration d'un environnement de développement pour \component{jbpmmicroscope} est expliquée
dans la page \href{https://intranet.genoscope.cns.fr/agc/redmine/projects/microscopeworkflow/wiki/JBPMprocedure}{[[JBPM procedure environnement de dev/declaration nouveau workflow]]}.

Une fois ceci fait, il faut compiler \component{jbpmmicroscope-client} avec le profil \texttt{config-microcloud}:
\begin{lstlisting}[style=bash]
$ mvn -Pconfig-microcloud -s ~/.m2/genosphere-settings.xml clean install
\end{lstlisting}
Cette commande crée le fichier \filename{jbpmmicroscope.jar} dans le dossier \path{jbpmmicroscope-client/target}.
Il faut ajouter les sources (qui sont nécessaires pour le déploiement des workflows dans les VM) et le script \script{JBPMmicroscope}:
\begin{lstlisting}[style=bash]
$ jar uf jbpmmicroscope-client/target/jbpmmicroscope.jar -C ./ src/
$ jar uf jbpmmicroscope-client/target/jbpmmicroscope.jar -C ./ JBPMmicroscope
\end{lstlisting}

Il faut compiler \component{jbpmmicroscope-server} avec le profil \texttt{config-microcloud}:
\begin{lstlisting}[style=bash]
$ mvn -Pconfig-microcloud -s ~/.m2/genosphere-settings.xml package
\end{lstlisting}
Cette commande crée les fichiers \filename{SystemActorsLauncher.jar} et \filename{jbpmmicroscope-server.war}
dans le dossier \path{jbpmmicroscope-server/target}.

Copier le .war et les .jar sur etna0 dans le répertoire \path{/env/cns/wwwext/html/agc/ftp/MicroCloud/}.
La VM master importe ces fichiers à l'aide de liens symboliques.

Le schéma de la base \DB{JBPMmicroscope} est créé automatiquement par hibernate (voir le paramètre \textbf{hbm2ddl.auto} dans \filename{hibernate.cfg.xml}) lors de la première exécution d'une commande \script{JBPMmicroscope}.

Les identifiants de connexion à la base \DB{JBPMmicroscope} (via hibernate) sont les suivants:
\begin{description}
    \item[login:] \texttt{jbpm}
    \item[password:] \texttt{jbpm}
\end{description}
Ces identifiants peuvent être modifiés dans le composant master, ils correspondent aux paramètres \textbf{jbpm\_user} et \textbf{jbpm\_password}.
Pour modifier ces identifiants, il faut modifier le fichier \path{hibernate.cfg.xml}, régénérer les .jar et le .war et les copier dans \path{/env/cns/wwwext/html/agc/ftp/MicroCloud}).\todo{Mettre dans partie technique}

\component{jbpmmicroscope} nécessite un serveur Tomcat pour fonctionner.
Tomcat 9 est installé dans le composant master : \path{/env/cns/proj/agc/tools/COMMON/JBPMmicroscope/tomcat}.
Les identifiants tomcat sont les suivants:
\begin{description}
    \item[login:] \texttt{tomcat}
    \item[password:] \texttt{tomcat}
\end{description}
Ces identifiants peuvent être modifiés dans le composant master, ils correspondent aux paramètres \textbf{tomcat\_user} et \textbf{tomcat\_password}.

\section{Mettre à jour la VM permanente}
\todo{Compléter cette partie + ajouter un lien au début}

Il faut penser à mettre à jour les tables de la VM permanente en fonction des besoins. Pour cela, s'inspirer du script \script{microscopeCreateDBschemas.py}, par exemple, et utiliser les fichiers de données ainsi générés.

Pour copier les données, mieux vaut utiliser rsync et copier les données dans le répertoire \textbf{/var/lib/mysql-files/} de la VM permanente avant insertion en base :
\begin{lstlisting}[style=bash]
$ LOAD DATA INFILE '/var/lib/mysql-files/$file.DB' INTO TABLE $table;
\end{lstlisting}
