\section{Informations logiciels}

\subsection{Tomcat}
Version de Tomcat installée : tomcat-9.0.31 \url {http://mirrors.ircam.fr/pub/apache/tomcat/tomcat-9/v9.0.31/bin/apache-tomcat-9.0.31.tar.gz}

\subsection{Pegasus-mpi-cluster}
Liste des release : \url{https://github.com/pegasus-isi/pegasus/releases}.
C'est la version 4.9.2 \url {https://github.com/pegasus-isi/pegasus/archive/4.9.2.zip} qui est installée dans MicroCloud (version 4.6.2 dans MicroScope).
La librairie libnuma pose problème lors de l'installation de pegasus-mpi-cluster mais celle-ci n'est pas nécessaire pour les architectures non NUMA (cf. master : \url{https://github.com/IFB-ElixirFr/biosphere-microcloud/blob/master/master/04_deployment.sh}).

\subsection{Bagsub}
Le script import\_modules.sh \url {https://github.com/IFB-ElixirFr/biosphere-microcloud/blob/master/master/import_modules.sh} modifie le module bagsub, en supprimant l'option ulimit qui ne fonctionne pas en Dash, et, en ajoutant le paramètre MCA btl\_tcp\_if\_exclude pour exclure les interfaces réseaux docker0 et lo qui peuvent engendrer des conflits lors du lancement de jobs slurm.

\subsection{JBPM}
Voir les documents :
\begin{itemize}
\item Bilan jBPM
 \url{https://docs.google.com/document/d/1jKhNCV0Po0t0Z1Zru7Qe2KnJA8_8yIjictccrYvGksU/edit#heading=h.7uty8jgvcw4i} 
\item Installation jBPM \url{https://docs.google.com/document/d/1kCW1xjsifGyQiZYWtgj2ul9jX6FdCF7xknBirS-gB5k/edit}{}.
\end{itemize}

\subsection{BioMAJ}

Nous avons implémenté un certain nombre de chose dans BioMAJ, notamment dans le cadre du projet PANGBANK :
\begin{itemize}
    \item Utilisation des hardlinks lors de la mise à jour d'une banque
    \item Ajout d'options SSL
    \item Refactoring des downloaders
    \item Retry plus intelligent lors des téléchargements
\end{itemize}
Actuellement nous n'avons pas suffisamment de place pour installer BioMaJ sur la VM permanente.
Certaines banques sont déjà accessibles depuis les autres cloud mais pas depuis prabi-girofle (qui nous donnée accès à la VM permanente).

\subsection{Conda}
À l'heure actuelle, \textbf{conda} ne me semble pas totalement adapté à nos besoins.
En effet, installer un environnement par WF est lourd (plusieurs centaines de Mo/environnement); l'autre possibilité est d'installer un environnement pour chaque groupe d'outils compatibles mais on doit gérer ça manuellement.
