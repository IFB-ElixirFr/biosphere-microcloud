\chapter{Informations logiciels}

\todo[inline]{Ce chapitre est-il utile ?}


\section{JBPM}

\begin{itemize}
\item Bilan jBPM : voir \url{https://docs.google.com/document/d/1jKhNCV0Po0t0Z1Zru7Qe2KnJA8_8yIjictccrYvGksU/edit#heading=h.7uty8jgvcw4i}
\item Installation jBPM : voir \url{https://docs.google.com/document/d/1kCW1xjsifGyQiZYWtgj2ul9jX6FdCF7xknBirS-gB5k/edit}
\end{itemize}

\section{Conda}
À l'heure actuelle, \textbf{conda} ne me semble pas totalement adapté à nos besoins.
En effet, installer un environnement par WF est lourd (plusieurs centaines de Mo/environnement); l'autre possibilité est d'installer un environnement pour chaque groupe d'outils compatibles mais on doit gérer ça manuellement.
