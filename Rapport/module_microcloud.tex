\chapter{Module \micWEBdeployVer} \label{chap:micwebdeploy}

Ce module contient les scripts utilisés côté Genoscope pour gérer les fichiers (logiciels et les données) nécessaires pour déployer MicroCloud (voir~\autoref{chap:creer_nouvelle_version}).
C'est une version de développement basée sur \module[1.0]{micWEBdeploy}.

Il est constitué des scripts suivants :
\begin{description}
    \item[\script{microscopeRelease.py}] : copie le code du web, compile le code JavaScript (avec \script{microscopeCompileJS.sh}), copie les schémas de certaines bases et les données minimales de certaines bases.
    \item[\script{microscopeCopyOid.py}] : permet de copier les données de l'organisme \theOrg{} (O\_id=\theOid)
    \item[\script{createModulesTarball.py}]: créé les archives des modules à importer dans MicroCloud.
\end{description}

Comme expliqué, ce module est très lié aux scripts de déploiement des VM.

\section{\script{microscopeRelease.py}}

Les schémas copiés sont : pkgdb, REFSEQDB, GO\_Conf, GO\_RES, GO\_CPD, PUB\_CPD et PRESTATIONDB.
\newline

Les données copiées sont : pkgdb (tables : Maintenance, Country, Amiga\_Params,
Annotator pour A\_name='guest', Sequence\_Checkpoint\_Desc et Sid\_Config), GO\_RES (tables : ORGCLUST\_clustering\_param et ORGCLUST\_distance\_param.
) et GO\_Conf (prendre les données de toutes les tables). Voir page wiki : \url{https://intranet.genoscope.cns.fr/agc/redmine/projects/microcloud/wiki/Tables_necessaires_a_installation}


\section{\script{createModulesTarball.py}}

Créé les liens symboliques dans \path{/env/cns/wwwext/html/agc/ftp/MicroCloud}.
Créé le fichier \path{modules.txt} dans \path{/env/cns/wwwext/html/agc/ftp/MicroCloud}.
\newline
Puis, la VM master importe les modules listés dans modules.txt (voir \url{https://github.com/IFB-ElixirFr/biosphere-microcloud/blob/master/master/import_modules.sh}).

\begin{lstlisting}[style=bash]
$ createModulesTarball.py --modules_list bagsub-2.4.3 AGCScriptToolMic-2.0 micGenome-7.0.0 micJBPMwrapper-3.10.8 micPrestation-2.0 micDirecton-1.0
\end{lstlisting}

\section{\script{microscopeCopyOid.py}}

\begin{mycolorbox}
    Pour le moment seules les données concernant l'O\_id \theOid{} peuvent être récupérées avec le script microscopeCopyOid.py car certains organismes ont besoin de tables spécifiques de la base GO\_SPE (typiquement la table acineto\_KO pour \theOrg{}).
\end{mycolorbox}

\begin{lstlisting}[style=bash]
$ Oid=\theOid{}
$ microscopeCopyOid.py --oid ${Oid} --output /env/cns/wwwext/html/agc/ftp/MicroCloud/microscope_${Oid}.tar.gz
\end{lstlisting}

\section{\script{microscopeCreateDBschemas.py}}
A permis de générer les .sql utilisés pour créer les schémas des bases de données dans la VM permanente.

\begin{lstlisting}[style=bash]
$ microscopeCreateDBschemas.py --output /env/cns/wwwext/html/agc/ftp/MicroCloud/DBschemas.tar.gz
\end{lstlisting}

\section{\script{taxonomyDBCopyTaxId.py}}
Permet de copier les données de la base TAXONOMYDB pour un tax\_id donné.

\begin{lstlisting}[style=bash]
$ taxonomyDBCopyTaxId.py --tax_id 202950 --output /env/cns/wwwext/html/agc/ftp/MicroCloud/taxonomyDBCopyTaxId.tar.gz
\end{lstlisting}

\section{Les scripts côté serveur}

Les scripts coté serveur importent et utilisent les fichiers générés par le module \micWEBdeployVer.
\bigskip

Les scripts concernant l'installation de MicroScope sont dans le dossier \textbf{frontend} \url{https://github.com/IFB-ElixirFr/biosphere-microcloud/blob/master/master} :
\begin{description}
    \item[\script{install\_microscope.sh}] Le frontend est basé sur Apache2, PHP7.1 et phpMyAdmin. L'installation est double : il faut d'abord configurer quelques bases (Apache2, PHP7.1 et phpMyAdmin) et monter le répertoire partagé après avoir configuré MicroScope.
    \item[\script{import\_Oid.sh}] permet de créer un utilisateur et un schéma MySQL et d'insérer les données de l'Oid \theOid{} (cela est nécessaire car le Web ne fonctionne pas sans données), et il permet aussi de copier les web\_data dans le répertoire approprié.
    \item[\script{create\_federated\_links.sh}] permet de créer des liens fédérés depuis la VM mysql vers la VM permanente.
\end{description}
\bigskip

Les scripts concernant l'installation de jBPM sont dans le dossier \textbf{master} \url{https://github.com/IFB-ElixirFr/biosphere-microcloud/blob/master/master}. Ces scripts sont décrits ci-dessous. Ils permettent de configurer l'environnement du moteur de workflows jBPM et de lancer un workflow simple (DIRECTON). L'ordonnancement des wokflows est ensuite géré à l'aide du cluster SLURM installé sur les VM master et slaves.

\begin{description}
    \item[\script{install\_jbpm.sh}] : fait l'installation du moteur de workflow jBPM et du server d'application Tomcat.
    \item[\script{config\_jbpm.sh}] : gère la création du schéma de la base de données JBPMmicroscope ainsi que les configurations de base.
    \item[\script{create\_user.sh}] : est utilisé pour créer un utilisateur MicroScope avoir à passer par l'interface.
    \item[\script{import\_modules.sh}] : permet d'importer et d'installer les modules de MicroScope (bagsub, AGCScriptToolMic, micGenome, micJBPMwrapper, micPrestation, micDirecton, ...).
\end{description}
