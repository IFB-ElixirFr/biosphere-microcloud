\section {Module MicroCloud}

Pour lancer le module MicroCloud :
\begin{lstlisting}[style=bash]
module load micWEBdeploy/0-MicroCloud-1.0
\end{lstlisting}

Répertoire d'export des archives et autres fichiers sur etna0 : \path{/env/cns/wwwext/html/agc/ftp/MicroCloud/}.

\subsection{microscopeRelease.py}
Copie le code du web.
\newline
Créé les schémas des bases pkgdb, REFSEQDB, GO\_Conf, GO\_RES, GO\_CPD, PUB\_CPD et PRESTATIONDB.
\newline
Copie les données minimales de pkgdb, GO\_RES et GO\_Conf.
\newline
Lance le script \textbf{microscopeCompileJS.sh} qui compile le code JS. Les scripts \textbf{instanceWEBdeploy.sh} et \textbf{microscopeWEBdeploy.sh} ont servis à écrire le script \textbf{microscopeCompileJS.sh}.

\begin{lstlisting}[style=bash]
microscopeRelease.py --version 3.14.0 --output /env/cns/wwwext/html/agc/ftp/MicroCloud/microcloud-3.14.0.tar.gz
\end{lstlisting}

\subsection{createModulesTarball.py}
Créé les archives des modules à importer dans MicroCloud.
\newline
Créé les liens symboliques dans \path{/env/cns/wwwext/html/agc/ftp/MicroCloud}
\newline
Créé le fichier \path{modules.txt} dans \path{/env/cns/wwwext/html/agc/ftp/MicroCloud}
\newline
Puis la VM master importe les modules listés dans modules.txt (cf. \href{https://github.com/IFB-ElixirFr/biosphere-microcloud/blob/master/master/import_modules.sh}{import\_modules.sh}).

\begin{lstlisting}[style=bash]
createModulesTarball.py --modules_list bagsub-2.4.3 AGCScriptToolMic-2.0 micGenome-7.0.0 micJBPMwrapper-3.10.8 micPrestation-2.0 micDirecton-1.0
\end{lstlisting}
Puis, on importe la liste des modules \url{https://github.com/IFB-ElixirFr/biosphere-microcloud/blob/master/master/import_modules.sh} dans la VM master.

\subsection{microscopeCopyOid.py}
Permet de copier les données de l'organisme \textit{Acinetobacter baylyi} O\_id=31. 

\begin{mycolorbox}
	Pour le moment seules les données concernant l'O\_id 31 peuvent être récupérées avec le script microscopeCopyOid.py car certains organismes ont besoin de tables spécifiques de la base GO\_SPE (typiquement la table acineto\_KO pour Acinetobacter baylyi).
\end{mycolorbox}

\begin{lstlisting}[style=bash]
microscopeCopyOid.py --oid 31 --output /env/cns/wwwext/html/agc/ftp/MicroCloud/microscope_31.tar.gz
\end{lstlisting}

\subsection{microscopeCreateDBschemas.py}
A permis de générer les .sql utilisés pour créer les schémas des bases de données dans la VM permanente.

\begin{lstlisting}[style=bash]
microscopeCreateDBschemas.py --output /env/cns/wwwext/html/agc/ftp/MicroCloud/DBschemas.tar.gz
\end{lstlisting}

\subsection{taxonomyDBCopyTaxId.py}
Permet de copier les données de la base TAXONOMYDB pour un tax\_id donné.

\begin{lstlisting}[style=bash]
taxonomyDBCopyTaxId.py --tax_id 202950 --output /env/cns/wwwext/html/agc/ftp/MicroCloud/taxonomyDBCopyTaxId.tar.gz
\end{lstlisting}