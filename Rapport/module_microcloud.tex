\chapter{Module \micWEBdeployVer} \label{chap:micwebdeploy}

Ce module contient les scripts utilisés côté Genoscope pour gérer les fichiers (logiciels et les données) nécessaires pour déployer MicroCloud (voir~\autoref{chap:creer_nouvelle_version}).
C'est une version de développement basée sur \module[1.0]{micWEBdeploy}.

Ce chapitre décrit les scripts ajoutés dans cette version.
Comme expliqué, ce module est très lié aux scripts de déploiement des VM.

\section{\script{microscopeRelease.py}}

Ce scripts crée une archive du code, du schéma et des données nécessaires pour le web
qui sera utilisée par \script{install\_microscope.sh} pour l'installation sur \SScomponent{frontend} (voir ~\autoref{frontend}).

Le code web est extrait à partir du dépôt subversion.
Le code Javascript est compilé avec \script{microscopeCompileJS.sh} pour ne pas devoir faire ça sur \SScomponent{frontend}.

Les schémas copiés sont : \DB{pkgdb}, \DB{REFSEQDB}, \DB{GO\_Conf}, \DB{GO\_RES}, \DB{GO\_CPD}, \DB{PUB\_CPD} et \DB{PRESTATIONDB}.
Les données copiées sont :
\begin{description}
    \item[\DB{pkgdb}:] données des tables \DBtable{Maintenance}, \DBtable{Country}, \DBtable{Amiga\_Params}, \DBtable{Annotator} (pour \variable{A\_name='guest'}),
    \DBtable{Sequence\_Checkpoint\_Desc} et \DBtable{Sid\_Config}.
    \item[\DB{GO\_RES}:] données des tables \DBtable{ORGCLUST\_clustering\_param} et \DBtable{ORGCLUST\_distance\_param}.
    \item[\DB{GO\_Conf}:] données de toutes les tables.
\end{description}

Voir page wiki \href{https://intranet.genoscope.cns.fr/agc/redmine/projects/microcloud/wiki/Tables_necessaires_a_installation}{[[Tables nécessaires à l'installation]]}.

\section{\script{microscopeCopyOid.py}}

Ce script permet de copier les données d'un organisme pour les importer dans une instance MicroCloud (voir \script{import\_Oid.sh}).

\begin{mycolorbox}
    Pour le moment seules les données concernant l'Oid \theOid{} peuvent être récupérées avec le script microscopeCopyOid.py car certains organismes ont besoin de tables spécifiques de la base GO\_SPE (typiquement la table acineto\_KO pour \theOrg{}).
\end{mycolorbox}

\section{\script{createModulesTarball.py}}

Ce script créé les archives des modules à importer dans MicroCloud:
\begin{itemize}
    \item Créé les liens symboliques dans \path{/env/cns/wwwext/html/agc/ftp/MicroCloud}.
    \item Créé le fichier \path{modules.txt} dans \path{/env/cns/wwwext/html/agc/ftp/MicroCloud}.
\end{itemize}

\section{\script{microscopeCreateDBschemas.py}}

Ce script dumpe les schémas des bases de données dans la VM permanente.

\begin{lstlisting}[style=bash]
microscopeCreateDBschemas.py --output /env/cns/wwwext/html/agc/ftp/MicroCloud/DBschemas.tar.gz
\end{lstlisting}

\section{\script{taxonomyDBCopyTaxId.py}}

Ce script dumpe les données de la base \DB{TAXONOMYDB} pour un \variable{tax\_id} donné.

\begin{lstlisting}[style=bash]
taxonomyDBCopyTaxId.py --tax_id (*\theTaxID*) --output /env/cns/wwwext/html/agc/ftp/MicroCloud/taxonomyDBCopyTaxId.tar.gz
\end{lstlisting}
