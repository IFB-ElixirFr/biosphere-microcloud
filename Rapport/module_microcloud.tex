\chapter{Module \micWEBdeployVer}

\todo[inline]{Ajouter une section sur les scripts côté serveur qui permettent d'installer le logiciel.}

Ce module contient les scripts utilisés pour gérer les logiciels et les données nécessaires pour déployer MicroCloud (voir~\autoref{chap:creer_nouvelle_version}).
C'est une version de développement basée sur \module[1.0]{micWEBdeploy}.

Il est constitué des scripts suivants:
\begin{description}
	\item[\script{microscopeRelease.py}]: copie le code du web, compile le code JavaScript (avec \script{microscopeCompileJS.sh}), copie les schémas de certaines bases et les données minimales de certaines bases.
	\item[\script{microscopeCopyOid.py}]: permet de copier les données de l'organisme \textit{Acinetobacter baylyi} (O\_id=31)
	\item[\script{createModulesTarball.py}]: créé les archives des modules à importer dans MicroCloud.
\end{description}

\section{\script{microscopeRelease.py}}

Les schémas copiés sont : pkgdb, REFSEQDB, GO\_Conf, GO\_RES, GO\_CPD, PUB\_CPD et PRESTATIONDB.
\newline

Les données copiées sont : pkgdb (tables : Maintenance, Country, Amiga\_Params,
Annotator pour A\_name='guest', Sequence\_Checkpoint\_Desc et Sid\_Config), GO\_RES (tables : ORGCLUST\_clustering\_param et ORGCLUST\_distance\_param.
) et GO\_Conf (prendre les données de toutes les tables). Voir page wiki : \url{https://intranet.genoscope.cns.fr/agc/redmine/projects/microcloud/wiki/Tables_necessaires_a_installation}


\section{\script{createModulesTarball.py}}

Créé les liens symboliques dans \path{/env/cns/wwwext/html/agc/ftp/MicroCloud}.
Créé le fichier \path{modules.txt} dans \path{/env/cns/wwwext/html/agc/ftp/MicroCloud}.
\newline
Puis, la VM master importe les modules listés dans modules.txt (voir \url{https://github.com/IFB-ElixirFr/biosphere-microcloud/blob/master/master/import_modules.sh}).

\begin{lstlisting}[style=bash]
createModulesTarball.py --modules_list bagsub-2.4.3 AGCScriptToolMic-2.0 micGenome-7.0.0 micJBPMwrapper-3.10.8 micPrestation-2.0 micDirecton-1.0
\end{lstlisting}

\section{\script{microscopeCopyOid.py}}

\begin{mycolorbox}
	Pour le moment seules les données concernant l'O\_id 31 peuvent être récupérées avec le script microscopeCopyOid.py car certains organismes ont besoin de tables spécifiques de la base GO\_SPE (typiquement la table acineto\_KO pour Acinetobacter baylyi).
\end{mycolorbox}

\begin{lstlisting}[style=bash]
microscopeCopyOid.py --oid 31 --output /env/cns/wwwext/html/agc/ftp/MicroCloud/microscope_31.tar.gz
\end{lstlisting}

\section{\script{microscopeCreateDBschemas.py}}
A permis de générer les .sql utilisés pour créer les schémas des bases de données dans la VM permanente.

\begin{lstlisting}[style=bash]
microscopeCreateDBschemas.py --output /env/cns/wwwext/html/agc/ftp/MicroCloud/DBschemas.tar.gz
\end{lstlisting}

\section{\script{taxonomyDBCopyTaxId.py}}
Permet de copier les données de la base TAXONOMYDB pour un tax\_id donné.

\begin{lstlisting}[style=bash]
taxonomyDBCopyTaxId.py --tax_id 202950 --output /env/cns/wwwext/html/agc/ftp/MicroCloud/taxonomyDBCopyTaxId.tar.gz
\end{lstlisting}

\section{Les scripts côtés server}

Ils sont accessibles à l'adresse : \url{https://github.com/IFB-ElixirFr/biosphere-microcloud/tree/master/frontend} et sont exécutés sur la VM frontend.
\begin{description}
	\item[\script{install\_microscope.sh}] :  ce script contient les commandes SQL permettant de créer les schémas des bases de données et d'insérer les données dans le serveur de la VM mysql. Il utilise les fichiers générés par le script microscopeRelease.py.
	\item[\script{create\_federated\_links.sh}] : ce script permet de créer des liens federated entre le serveur de la VM mysql et le serveur de la VM permanente.
	\item[\script{import\_Oid.sh}] : ce script sert à insérer les données récupérées à l'aide du script microscopeCopyOid.py. Il permet de copier les web\_data dans le répertoire approprié.
\end{description}

