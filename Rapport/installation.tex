\section {Installation et déploiement des VM}

\subsection{Déployer l'appliance MicroCloud}

Pour déployer l'appliance MicroCloud, il faut aller sur le site  \href{https://biosphere.france-bioinformatique.fr/catalogue/}{RAINBio}, choisir l’\href{https://biosphere.france-bioinformatique.fr/catalogue/appliance/150/}{Appliance MicroCloud} et cliquer sur \textbf{Lancer} ou \textbf{Déploiement avancé}.

Les déploiements se font sur le cloud prabi-girofle sur lequel est installé la \hyperref[VM permanente]{VM permanente}. Un déploiement prend environ 40 minutes. Il vaut mieux en lancer 2 en même temps afin d'être sûr d'en avoir au moins un de fonctionnel.
Les échecs dûs au paquet \textbf{procps} sont assez fréquents.

\subsection {Description des VM}

Il y 6 machines virtuelles (VM) au total dont 5 conçues depuis Nuvla et 1 machine OpenStack déployée sur le cloud prabi-girofle. Le code des VM est dans le dépôt biosphere-microcloud (\url{https://github.com/IFB-ElixirFr/biosphere-microcloud/}) et il y a un dossier par VM:

\begin{itemize}
	\item frontend voir section \ref{frontend}
	\item mysql voir section \ref{mysql}
	\item VM permanente voir section \ref{VMpermanente}
	\item master et slave voir section \ref{master&slave}
	\item nfsserver voir section \ref{nfsserver}

\end{itemize}
Il y a 2 serveurs de bases de données dans MicroCloud. Un serveur mysql installé via Docker dans la VM mysql et un serveur mysql sur la VM permanente. Pour se connecter aux VM voir la doc : \url{https://intranet.genoscope.cns.fr/agc/redmine/projects/microcloud/wiki/Connexion_aux_VM}

\subsubsection {VM frontend}

\label{frontend} La VM frontend permet de déployer la partie web de la plateforme MicroScope. L'image de base est une image CentOS 7.\\
Le code est dans le dossier frontend voir \url{https://github.com/IFB-ElixirFr/biosphere-microcloud/tree/master/frontend/}.
Le composant slipstream est dans le dossier frontend voir \url{https://nuv.la/module/ifb/devzone/MicroCloud/frontend/}.

Lors du déploiement de la VM le code web copié dans le répertoire \textbf{/var/www/html/} de la VM.
L'URL de la plateforme est : \textbf{\$IP\_frontend/home/index.php}\\
La VM frontend possède un client mariaDB (non pas MySQL du fait de conflits existants entre le dépôt remi-php71 installé et le dépôt IUS qui fournit le client MySQL).
\newline

Pour se connecter en ssh à la VM frontend : 
\begin{lstlisting}[style=Bash]
ssh centos@${IP_frontend}
\end{lstlisting}

\begin{mycolorbox}Si le message d’erreur \textbf{256} s’affiche, cela signifie simplement qu’il n’y a pas d’organisme en base. De ce fait, la plupart des onglets sont inaccessibles ainsi que le formulaire d’authentification.
\end{mycolorbox}

\subsubsection {VM mysql}

\label{mysql}
Le code est dans le dossier mysql voir \url{https://github.com/IFB-ElixirFr/biosphere-microcloud/tree/master/mysql/}.
Le composant slipstream est dans le dossier mysql voir \url{https://nuv.la/module/ifb/devzone/MicroCloud/mysql/}.

L'image de base est une image CentOS 7.
La VM mysql possède un serveur MySQL installé via Docker sur lequel on retrouve les bases de données pkgdb, GO\_CPD, GO\_Conf, GO\_RES, PUB\_CPD, REFSEQDB, JBPMmicroscope et PRESTATIONDB.
\newline
Les tables nécessaires à l'installation de MicroScope sont également listées \href{https://intranet.genoscope.cns.fr/agc/redmine/projects/microcloud/wiki/Tables_necessaires_a_installation}{ici}.
\newline

Pour se connecter à la VM mysql, il faut passer par le frontend :
\begin{lstlisting}[style=bash]
ssh -A centos@${IP_mysql} -J centos@${IP_frontend}
\end{lstlisting}
\bigskip

Si le serveur ne réponds pas, il faut aller voir si le docker n'a pas planté (cela arrive pour des requêtes SQL trop gourmandes en RAM). Pour relancer le docker :
\begin{lstlisting}[style=bash]
sudo su
docker ps -a
docker start ${ID_container}
\end{lstlisting}

\subsubsection {VM permanente}

\label{VMpermanente} C'est une VM OpenStack (\textbf{umr5558-microcloud.univ-lyon1.fr}) du cloud prabi-girofle disposant de 200 GO de stockage et 8 GO de RAM (elle est actuellement sous-dimensionnée par rapport à nos besoins). La machine a été installée avec Sylvain Bonneval. L'image de base est une image Debian 9.8.\\
La machine permanente n'est accessible depuis l'extérieur qu'en SSH donc nous ne pouvons pas y accéder en MySQL (port 3306) depuis un autre cloud.
\newline

La procédure d'installation est dans le fichier \textbf{Installation.md} du répertoire \textbf{/root}. La VM permanente sert au stockage des données des banques. Nous avons utilisé \textbf{rsync} pour l'import des données dans le serveur MySQL.
\newline
Logiciels installés : serveur mysql, rsync, phpMyAdmin (installé mais non configuré). 
\newline

A terme, il serait utile d'avoir un système de gestion des banques tel que \href{https://biomaj.genouest.org/}{BioMaJ} installé sur cette machine.\\

Pour se connecter : 
\begin{lstlisting}[style=bash]
ssh root@134.214.33.214
\end{lstlisting}
\bigskip

Sur le serveur MySQL, il y a les schémas des bases DB et une partie des données de la base pour les tests (nous avons testés les onglets \textbf{Genome Browser}, \textbf{Identical Gene Names}) :
\begin{itemize}
	\item les bases ANTISMASHDB, CARDDB, COGDB, EGGNOGDB, ENZYMEDB, ESSDB, FIGFAMDB, INTERPRODATADB, KEGGDB, RHEADB, TAXONOMYDB, TIGRFAMDB, UNIFIREDB, UNIPROTKBDB, VIRULENCEDB, microcyc, DBWorkflow
	\item les données (en partie) de la base UNIPROTKBDB pour les tests
	\item les données de DBWORKFLOW
	\item les données de l'organisme \textit{Acinetobacter} dans la base TAXONOMYDB (taxon id 202950)
\end{itemize}

Le script \textbf{microscopeDBschema.py} du module MicroCloud a été utilisé pour créer les schémas.
\newline

Pour se connecter au serveur MySQL :
\begin{lstlisting}[style=bash]
mysql -p{MOT_DE_PASSE_SERVEUR_MYSQL_VM_PERMANENTE}
\end{lstlisting}
\bigskip

Les mots de passe (serveur mySQL et phpMyAdmin) sont stockés dans le répertoire \textbf{/root} de la VM.
\newline 

Pour copier les données :
Mieux vaut utiliser rsync et copier les données dans le répertoire \textbf{/var/lib/mysql-files/} de la VM permanente avant insertion en base.\\

\begin{lstlisting}[style=bash]
LOAD DATA INFILE '/var/lib/mysql-files/$file.DB' INTO TABLE $table;
\end{lstlisting}

\subsubsection{VM master et VM slave(s)}
\label{master&slave} Il y a un cluster Slurm installé sur ces VM pour faire tourner les WF. Le code d’installation de ces VM est dans le dépôt \href{https://github.com/IFB-ElixirFr/biosphere-commons}{biosphere-commons}.
L'image de base est une image Ubuntu 18.
Le code est basé sur les travaux de Jonathan Lorenzo et Bryan Brancotte.
Les composants slurm master et slaves ont été copiés depuis l'appliance Slurm\_Cluster\_ubuntu18 voir \url{https://nuv.la/module/ifb/devzone/jlorenzo/cluster/Slurm_Cluster_ubuntu18}

Concernant la VM master, le code est dans le dossier master voir
\url{https://github.com/IFB-ElixirFr/biosphere-microcloud/tree/master/master/}.
Le code du composant slipstream est dans le dossier master voir \url{https://nuv.la/module/ifb/devzone/MicroCloud/master/}.

Pour la VM slave voir les dossiers correspondants.\\

URL Tomcat : \textbf{\$IP\_master}

\subsection{Lien federated entre la VM mysql et  la VM permanente}
Les bases DB des banques sont accessibles depuis la VM mysql via des liens \href{https://dev.mysql.com/doc/refman/8.0/en/federated-storage-engine.html}{federated} (frontend : \href{https://github.com/IFB-ElixirFr/biosphere-microcloud/blob/master/frontend/create_federated_links.sh}{create\_federated\_links.sh}).\\

\begin{mycolorbox}Lorsque les schémas des bases DB de la VM permanente sont mis-à-jour; il faut penser à redéployer MicroCloud pour que les liens federated soient également mis-à-jour.
\end{mycolorbox}

\subsection{Partage NFS entre les différentes VM}

\label{nfsserver} Le composant nfsserver a été créé en s'inspirant du travail de Stéphane Delmotte. Il permet de fournir un serveur partagé entre les différentes VM.
L'image de base est une image CentOS 7.
Le code est dans le dossier nfsserver voir  \url{https://github.com/IFB-ElixirFr/biosphere-microcloud/tree/master/nfsserver/}.
Le composant slipstream est dans le dossier nfsserver voir
\url{https://nuv.la/module/ifb/devzone/MicroCloud/nfsserver/}.

Le répertoire partagé \textbf{/var/nfsshare} du serveur NFS est monté dans le répertoire \textbf{/env}
des VM frontend, backend, master et slave(s). Le répertoire \textbf{/env} se veut similaire au répertoire de même nom sur l’etna0. 
Les fonctions permettant la création du répertoire partagé sont dans le fichier \href{https://github.com/IFB-ElixirFr/biosphere-microcloud/blob/master/lib.sh}{lib.sh}.
