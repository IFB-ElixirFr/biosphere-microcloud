\section{VM master}

\subsection{Importation des modules}
Faire la liste de tous les modules nécessaires à la création de la nouvelle instance MicroCloud.
Utiliser le script \textbf{createModulesTarball.py} du module MicroCloud.

Mise à jour de l'environnement des modules :
actuellement l'environnement de MicroCloud ne permet d'utiliser que les modules \textbf{bagsub-2.4.3}, \textbf{AGCScriptToolMic-2.0}, \textbf{micGenome-7.0.0}, \textbf{micJBPMwrapper-3.10.8}, \textbf{micPrestation-2.0} et \textbf{micDirecton-1.0}.
\newline

\begin{mycolorbox}
	Pour tout ajout de nouveaux modules, il faut mettre à jour le fichier \path{microcloud.profile} et créer les répertoires nécessaires (master : \href{https://github.com/IFB-ElixirFr/biosphere-microcloud/blob/master/master/04_deployment.sh}{04\_deployment.sh}).
\end{mycolorbox}

Puis sourcer le fichier \path{microcloud.profile} dans la VM master :
\begin{lstlisting}[style=bash]
source /env/cns/proj/agc/module/profiles/microcloud.profile
\end{lstlisting}
